%----------------------------------------------------------------------------------------
%	PACKAGES AND OTHER DOCUMENT CONFIGURATIONS
%----------------------------------------------------------------------------------------


\documentclass[margin, centered]{res}
\topmargin=-0.5in
\oddsidemargin -.5in
\evensidemargin -.5in
\textwidth=6.5in
\itemsep=0in
\parsep=0in
\newsectionwidth{1.05in}
\usepackage[pdftex]{graphicx}
\usepackage{etaremune}
\usepackage{enumitem}
\usepackage{wrapfig}
%\usepackage[dvipsnames]{xcolor}
\usepackage{helvet}
\usepackage{comment}
\usepackage{fancyhdr}
\usepackage{multicol}

\pagestyle{fancy}
\renewcommand{\headrulewidth}{0pt}
\fancyhf{}
\fancyfoot[CE,CO]{\footnotesize Pranav Satheesh ~\textbullet~ Curriculum Vitae}
\fancyfoot[LE,LO]{\footnotesize Dec 2021}
\fancyfoot[RE,RO]{\footnotesize \thepage}

\usepackage[svgnames]{xcolor}
\definecolor{C2}{RGB}{137, 6, 32}
\definecolor{C1}{RGB}{0, 109, 119}
\definecolor{C3}{RGB}{8, 76, 97}

\usepackage[colorlinks = true,
            linkcolor = C1,
            urlcolor  = C3,
            citecolor = C1,
            anchorcolor = C1]{hyperref}
\setlength{\textwidth}{6.5in} % Text width of the document
\setlength{\textheight}{720pt}



\begin{document}

%----------------------------------------------------------------------------------------
%	NAME AND ADDRESS SECTION
%----------------------------------------------------------------------------------------\
\begin{center}
    \hspace{-\hoffset}
    \vspace{3mm}
    \huge {\textcolor{black}{\textbf{Pranav Satheesh}}}\\
    
    \hspace{-\hoffset}
    \large \href{mailto:pranavsatheesh@ufl.edu}{pranavsatheesh@ufl.edu} ~\textbullet~ \href{https://pranav-satheesh.github.io/}{pranav-satheesh.github.io} \\
    \hspace{-\hoffset}
    Department of Physics, University of Florida
\end{center}

\vspace{-4mm}
\moveleft\hoffset\vbox{\hrule width 19cm height 1pt}
\vspace{-7mm}
\begin{resume}

%----------------------------------------------------------------------------------------
%	EDUCATION SECTION
%----------------------------------------------------------------------------------------
\section{Reserach Interests}
Supermassive black hole (SMBH) dynamics, Gravitational Waves, Formation of SMBHs and Galaxy evolution, AGNs

\section{Education}
\textbf{\href{https://www.iitm.ac.in/}{University of Florida}} \hfill 2022 - Present \\
\textbf{Ph.D.} in Physics 

\textbf{\href{https://www.iitm.ac.in/}{Indian Institute of Technology Madras}} \hfill 2017 - 2022 \\
\textbf{BS-MS Dual Degree} in Physics\\
\emph{Thesis: Modelling subdominant harmonic modes
of eccentric binary black hole waveforms}
\section{Employment}
\textbf{Research Assistant}\hfill 2023 - Present \\
\textbf{\href{https://www.iitm.ac.in/}{University of Florida}}\\
\emph{Mentored by Prof. Laura Blecha}

% \textbf{Teaching Assistant}\hfill 2023 - 2024 \\
% \textbf{\href{https://www.iitm.ac.in/}{University of Florida}}\\

% \textbf{Teaching Assistant}\hfill 2022 \\
% \textbf{\href{https://www.iitm.ac.in/}{Indian Institute of Technology Madras}} 


%----------------------------------------------------------------------------------------
%	EXPERIENCE SECTION
%----------------------------------------------------------------------------------------


\section{Fellowships and Awards}
\begin{itemize}[leftmargin=*]
 \item UF astrophysics Fellowship
 \item {\color{C2} Graduate fellowship} for first-year graduate students from UF Physics \hfill 2022 - 2023
 \item {\color{C2} 62nd Institute day} award for academic performance in Physics from IIT Madras \hfill 2021
 \item Selected among top 8 students in India for \href{https://swissnex.org/india/thinkswiss/}{ThinkSwiss Research Scholarship} \hfill 2020

 \item Recepient of the \href{http://www.inspire-dst.gov.in/scholarship.html}{INSPIRE-DST Scholarship for Higher Education}  \hfill 2017 - \textit{Present}
\end{itemize}


\section{Talks and Posters}

% \textbf{Contributed talks and posters}\\
\begin{itemize}[leftmargin=*]
    \item (Talk) Midwest Relativity meeting, University of Chicago
    \item \textbf{Pranav Satheesh}, Shashank Gandhi, Chandra Kant Mishra, {\color{C1} \textbf{6th IIT Madras physics in-house symposium, April 2022}}, {\color{C2} \emph{Parameter Estimation of Eccentric Binaries using a Frequency Domain Inspiral Waveform}}\\
    \item \textbf{Pranav Satheesh}, Shashank Gandhi, Chandra Kant Mishra, {\color{C1} \textbf{LIGO-Virgo-KAGRA collaboration meeting, March 2022}}, {\color{C2} \emph{Fisher analysis of eccentric binaries with higher mode frequency domain inspirals}}\\
    \item (Contributed poster)Tamal RoyChowdhury, Abhishek Chattaraj, \textbf{Pranav Satheesh}, Chandra Kant Mishra, \href{https://www.amaldi14.org/}{\textbf{14th Amaldi 2021, 19-23 July (online)}}, \href{https://drive.google.com/file/d/1tCGgniOafmLfrbhD4X7ca3VkhOJtOSc2/view?usp=sharing}{\color{C2} \textit{Elements of modelling binary black holes in eccentric orbits through inspiral, merger and ringdown stages}}
    \item (Poster) \textbf{Pranav Satheesh}, Prasenjit Saha, Hans Martin Schmid , \href{https://aas.org/meetings/aas237}{\textbf{237th American Astronomical Society meet, 2021}}, \href{aas237-aas.ipostersessions.com/Default.aspx?s=79-64-0C-43-B0-53-8B-48-C7-A1-41-CE-DF-A9-70-2A}{\color{C2}\textit{A spectropolarimetric method for predicting the gravitational wave polarization of LISA verification binaries}}
    \item  (Poster) \textbf{Pranav Satheesh}, \href{https://ras.ac.uk/ras-2020-posters}{\textbf{RAS Career Poster Exhibition, 2020}}, \href{https://ras.ac.uk/poster-contest/pranav-satheesh}{\color{C2} {\textit{Frequency Domain Gravitational Waveform Modelling for Eccentric Black Hole Binaries}}}\\ 
\end{itemize}   

\section{Research Experience}

\textbf{Studying merger outcomes of triple massive black holes systems } \hfill Feb 2022 - \emph{Present}\\
\emph{Advisor: \href{http://www.phys.ufl.edu/~lblecha/}{Dr. Laura Blecha}}
\vspace{0.1 cm}\\
Studying the outcomes of triple massive black hole systems in cosmological simulations. We can get an interacting triple system in certain cases of galaxy mergers. My work involves characterizing these triples in cosmological simulation and studying their merger outcomes and merger rate.  

\textbf{Modelling higher-order modes from eccentric Binary Black Hole mergers} \hfill Jul 2021 - Jul 2022\\
\emph{Advisors:  \href{https://www.icts.res.in/people/prayush-kumar}{Dr. Prayush Kumar, ICTS-TIFR} and \href{https://physics.iitm.ac.in/ckm}{Dr. Chandra Kant Mishra, IIT Madras}} 
\vspace{0.1 cm}\\
    Worked on an Inspiral-Merger-Ringdown gravitational waveform model
    for binary black holes in eccentric orbits known as \href{https://journals.aps.org/prd/abstract/10.1103/PhysRevD.97.024031}{ENIGMA}. My work involves extending the waveform from to
    include higher order modes that will play a crucial role in the search for eccentric binaries in future gravitational wave searches.


\textbf{Ready-to-use frequency domain waveform model for eccentric } \hfill Aug 2019 - Sep 2021 \\ \textbf{binary black holes including non-quadrupole modes}  \\ 
\emph{Advisor: \href{https://physics.iitm.ac.in/ckm}{Dr. Chandra Kant Mishra, IIT Madras}}
\vspace{0.1 cm}\\
Developing a ready-to-use frequency domain waveform model for eccentric binary black holes that includes non-quadrupole terms and considers periastron effects. The waveform will be used to
construcr an Inspiral-Merger-Ringdown waveform model in frequency domain.


\textbf{Polarimetric method for predicting gravitational wave polarization of}\hfill May 2020 - Aug 2020 \\ \textbf{LISA verification binaries}\\
\emph{Advisor: \href{https://www.ctac.uzh.ch/en/Research/research-groups/Prasenjit-Saha.html}{Prof. Prasenjit Saha, University of Zurich} }
\vspace{0.1 cm}\\
Developed a method utilizing Polarimetry to measure the orientation and inclination of the binary system (HP Lib). Such binaries are sure candidates for the  Laser Interferometer Space Antenna (LISA) mission.
My work was presented at the \href{https://aas.org/meetings/aas237}{\color{C2}237th American Astronomical Society meeting}.


\textbf{Studying primordial gravitational waves from inflation and reheating phase} \hfill Aug 2021 - Present \\
\emph{Advisor: \href{https://physics.iitm.ac.in/~sriram/index.html}{Prof. L. Sriramkumar , IIT Madras}}
\vspace{0.1 cm}\\
Studying the evolution of primordial gravitational waves during the inflationary era data and the reheating phase of the universe. 






\textbf{Signal detection and parameter estimation using LIGO O1 and O2 data}\hfill May 2019 - Jul 2019 \\
\emph{Advisor: \href{https://www.iiserkol.ac.in/~rajesh/}{Prof. Rajesh Nayak , IISER Kolkata}}
\vspace{0.1 cm}\\
The project involved learning the basics of gravitational waves data analysis and parameter estimation using LIGO's publicly available data from O1 and O2 run.






\section{Publications}
\begin{itemize}[leftmargin=*]
    %\item (In preperation) \textbf{Pranav Satheesh},Chandra Kant Mishra\\
    %\textit{Ready-to-use eccentric frequency domain templates with non quadrapole} modes
    \item (In preperation) 
\end{itemize}




% \textbf{Attended meetings}\\
% \begin{itemize}[leftmargin=*]
%     \item BitGrav meeting %add poster link
%     \item CMU meeting
% \end{itemize}


\section{Professional Memberships}
\begin{itemize}[leftmargin=*]
    \item \textit{Associate Member}, \textbf{\color{C3} NanoGrav Collaboration} \hfill 2023-Present
    \item \textit{Graduate Member}, \textbf{\color{C3} American Astronomical Society} \hfill 2023-Present
    \item \textit{Graduate Member}, \textbf{\color{C3} American Physical Society} \hfill 2022-2023
    \item \textit{Member}, \textbf{\color{C3} LIGO Scientific Collaboration} \hfill 2021 -   2022
    \item \textit{Undergraduate Member}, \textbf{\color{C3} American Astronomical Society} \hfill 2020-2021
\end{itemize}

\section{Teaching Experience}
\begin{itemize}[leftmargin=*]
    \item {\color{C2} Teaching Assistant} Code/Astro workshop 2023
    \item {\color{C2} Teaching Assistant} UF first year labs
    \item {\color{C2} Teaching Assistant}, \textbf{\color{C3}Complex Networks (ID5080)} \hfill Aug 2021 - \textit{Present} \\
    \emph{Graduate level course at IIT Madras} 
    \item {\color{C2} Teaching Assistant}, \href{https://semaphorep.github.io/codeastro/}{\textbf{Code Astro 2021}}   \hfill June 2021  \\
    \emph{Virtual Software Engineering Workshop for Astronomy supported by \\ the Heising-Simons Foundation.}
\end{itemize}

\section{Other workshops and meetings}
\begin{itemize}[leftmargin=*]
    \item NanoGrav meeting online
    \item GW summer school 2022
    \item \href{https://youtu.be/zXDrQ_-WNUg}{\textbf{LISC Continous Gravitational Wave Workshop}} (Online) \hfill Oct 2021
    \item \href{https://sites.psu.edu/paxvii/}{\textbf{Physics and Astrophysics at the Extreme (PAX-VII) Workshop}} (Online) \hfill Aug 2021
    %\item Participant, \textbf{2021 Sagan Exoplanet Summer Virtual Workshop}, NASA Exoplanet Science Institute, California Institute of Technology, July 2021 
	\item \href{https://www.icts.res.in/program/gws2021}{\textbf{ICTS Summer School on Gravitational Wave Astronomy}} (Online) \hfill Jul 2021 
	\item \href{http://ipta4gw.org/meetings/2021/}{\textbf{IPTA Student Workshop}} (Online) \hfill June 2021 
	\item \href{https://www.icts.res.in/program/peu}{\textbf{Physics of the Early Universe}}, ICTS (Online)\hfill  Sep 2020 
	\item \href{https://www.icts.res.in/program/gws2020}{\textbf{ICTS Summer School on Gravitational Wave Astrophysics}} \hfill May 2020 
\end{itemize}

% \section{Relevant Coursework}
% General Relativity and Cosmology, Advanced General Relativity, Methods of Computational Physics, Numerical Methods and Programming lab, Classical Field Theory, Advanced Particle Physics, High Energy Physics, Statistical Physics, Quantum Mechanics, Classical Mechanics, 
% Mathematical Physics, Differential Equations

% \section{Outreach Talks}
% \begin{itemize}[leftmargin=*]
% 	\item  \textit{The Whats, Whys and Hows of Gravitational-wave Astronomy}, \textbf{BMS College of Engineering, Bengaluru}, November 2019
% 	\item \textit{Gravitational Waves - A New Tool for Cosmology!} at \textbf{Vigyan Samagam}, Visvesvaraya Industrial and Technological Museum, Bengaluru, India, August 2019

% \end{itemize}




% %----------------------------------------------------------------------------------------
% %	TECHNICAL SKILLS SECTION
% %----------------------------------------------------------------------------------------





\section{Technical \hspace{2mm} Skills}
\textbf{Programming Languages} - Python, C, C++, Shell script\\
\textbf{Softwares} - Mathematica, SAO DS9 \\
\textbf{Tools/Frameworks} - \LaTeX, Git



\section{Outreach}
\textbf{Author}
Astrobites \\

\textbf{Organizer}
Physics Graduate Committe, UF\\


\textbf{Service}
\begin{itemize}[leftmargin=*]
    \item Head, \href{https://horizoniitm.github.io/horizon/}{\textbf{Horizon: The Physics and Astronomy Club of IIT Madras}} \hfill 2019-2020\\
     I headed the student run physics and astronomy club at IIT Madras under the Center of Innovation (CFI).
    We engage the student community in the campus though various projects, lectures, workshops and compeetitive events.
\end{itemize}
\textbf{Articles}
\begin{itemize}[leftmargin=*]
    \item Undergraduate Research summary in \href{https://astrobites.org/2021/06/20/ur-a-spectropolarimetric-method-for-predicting-the-gravitational-wave-polarisation-of-lisa-verification-binaries/}{Astrobites} \\ \textit{\color{C2} UR: A spectropolarimetric method for predicting the gravitational wave polarisation of LISA verification }
\end{itemize}

\section{Science-communication pieces}
\begin{itemize}[leftmargin=*]
    \item \textit{\color{C2} Python for Astronomy}, An \href{https://youtu.be/HfYR0uwYAyM}{Youtube lecture series} offered by me as part of Horizon \hfill Jul 2020
    \item \textit{\color{C2} Relativity and Gravitation}, \href{https://github.com/HorizonIITM/summer-school-2021}{Horizon-IITM Summer School} \hfill July 2021
    \item Tutor, \textit{\color{C2} Analysis of Globular Clusters Using
    Colour-Magnitude Diagrams}, Shaastra IITM \hfill Jan 2020
\end{itemize}


\section{Public Talks}
\begin{itemize}[leftmargin=*]
    \item \textit{\color{C2} Python for Astronomy}, An \href{https://youtu.be/HfYR0uwYAyM}{Youtube lecture series} offered by me as part of Horizon \hfill Jul 2020
    \item \textit{\color{C2} Relativity and Gravitation}, \href{https://github.com/HorizonIITM/summer-school-2021}{Horizon-IITM Summer School} \hfill July 2021
    \item Tutor, \textit{\color{C2} Analysis of Globular Clusters Using
    Colour-Magnitude Diagrams}, Shaastra IITM \hfill Jan 2020
\end{itemize}





%\textbf{Talks}
% \begin{itemize}
%     \item GW
% \end{itemize}

% \section{References}
% \begin{itemize}[leftmargin=*]
%  \item Prof. Parameswaran Ajith, ICTS -- \href{mailto:ajith@icts.res.in}{ajith@icts.res.in}
%  \item Dr. Shasvath Kapadia, ICTS -- \href{mailto:shasvath.kapadia@icts.res.in}{shasvath.kapadia@icts.res.in}
%  \item Dr. Sumit Kumar, AEI Hannover -- \href{mailto:sumit.kumar@aei.mpg.de}{sumit.kumar@aei.mpg.de}
%  \item Prof. Bala Iyer, ICTS -- \href{mailto:bala.iyer@icts.res.in}{bala.iyer@icts.res.in}
% \end{itemize}

\end{resume}
\end{document}
