%----------------------------------------------------------------------------------------
%	PACKAGES AND OTHER DOCUMENT CONFIGURATIONS
%----------------------------------------------------------------------------------------


\documentclass[margin, centered]{res}
\topmargin=-0.5in
\oddsidemargin -.5in
\evensidemargin -.5in
\textwidth=6.5in
\itemsep=0in
\parsep=0in
\newsectionwidth{1.05in}
\usepackage[pdftex]{graphicx}
\usepackage{etaremune}
\usepackage{enumitem}
\usepackage{wrapfig}
%\usepackage[dvipsnames]{xcolor}
\usepackage{helvet}
\usepackage{comment}
\usepackage{fancyhdr}
\usepackage{multicol}

\pagestyle{fancy}
\renewcommand{\headrulewidth}{0pt}
\fancyhf{}
\fancyfoot[CE,CO]{\footnotesize Pranav Satheesh ~\textbullet~ Curriculum Vitae}
\fancyfoot[LE,LO]{\footnotesize Dec 2021}
\fancyfoot[RE,RO]{\footnotesize \thepage}

\usepackage[svgnames]{xcolor}
\definecolor{C2}{RGB}{137, 6, 32}
\definecolor{C1}{RGB}{0, 109, 119}
\definecolor{C3}{RGB}{8, 76, 97}

\usepackage[colorlinks = true,
            linkcolor = C1,
            urlcolor  = C3,
            citecolor = C1,
            anchorcolor = C1]{hyperref}
\setlength{\textwidth}{6.5in} % Text width of the document
\setlength{\textheight}{720pt}



\begin{document}

%----------------------------------------------------------------------------------------
%	NAME AND ADDRESS SECTION
%----------------------------------------------------------------------------------------\
\begin{center}
    \hspace{-\hoffset}
    \vspace{3mm}
    \huge {\textcolor{black}{\textbf{Pranav Satheesh}}}\\
    
    \hspace{-\hoffset}
    \large \href{mailto:pranavsatheesh17@gmail.com}{pranavsatheesh17@gmail.com} ~\textbullet~ \href{https://pranav-satheesh.github.io/}{pranav-satheesh.github.io} \\
    \hspace{-\hoffset}
    Indian Institute of Technology Madras, India
\end{center}

\vspace{-4mm}
\moveleft\hoffset\vbox{\hrule width 19cm height 1pt}
\vspace{-7mm}
\begin{resume}

%----------------------------------------------------------------------------------------
%	EDUCATION SECTION
%----------------------------------------------------------------------------------------
\section{Reserach Interests}
Gravitational Wave Astronomy and Astrophysics, Formation and evolution of compact objects, Post-Newtonian theory and Numerical Relativity, Tests of General Relativity using Gravitational Waves.
\\
\section{Education}


\textbf{\href{https://www.iitm.ac.in/}{Indian Institute of Technology Madras, Chennai, India}} \hfill 2017 - 2022 (expected) \\
\textbf{BS-MS Dual Degree Physics} \\
CGPA: 9.19/10\\
Physics GPA: 9.44/10
%----------------------------------------------------------------------------------------
%	EXPERIENCE SECTION
%----------------------------------------------------------------------------------------
\section{Research Experience}


\textbf{Modelling higher-order modes from eccentric Binary Black Hole mergers} \hfill Jul 2021 - Present\\
\emph{Advisors:  \href{https://www.icts.res.in/people/prayush-kumar}{Dr. Prayush Kumar, ICTS-TIFR} and \href{https://physics.iitm.ac.in/ckm}{Dr. Chandra Kant Mishra, IIT Madras}} 
\vspace{0.1 cm}\\
    Working on improving an Inspiral-Merger-Ringdown gravitational waveform model
    for binary black holes in eccentric orbits known as \href{https://journals.aps.org/prd/abstract/10.1103/PhysRevD.97.024031}{ENIGMA}. My work involves extending the waveform from to
    include higher order modes that will play a crucial role in the search for eccentric binaries in future gravitational wave searches.


\textbf{Ready-to-use frequency domain waveform model for eccentric } \hfill Aug 2019 - Sep 2021 \\ \textbf{binary black holes including non-quadrupole modes}  \\ 
\emph{Advisor: \href{https://physics.iitm.ac.in/ckm}{Dr. Chandra Kant Mishra, IIT Madras}}
\vspace{0.1 cm}\\
Developing a ready-to-use frequency domain waveform model for eccentric binary black holes that includes non-quadrupole terms and considers periastron effects. The waveform will be used to
construcr an Inspiral-Merger-Ringdown waveform model in frequency domain.


\textbf{Polarimetric method for predicting gravitational wave polarization of}\hfill May 2020 - Aug 2020 \\ \textbf{LISA verification binaries}\\
\emph{Advisor: \href{https://www.ctac.uzh.ch/en/Research/research-groups/Prasenjit-Saha.html}{Prof. Prasenjit Saha, University of Zurich} }
\vspace{0.1 cm}\\
Developed a method utilizing Polarimetry to measure the orientation and inclination of the binary system (HP Lib). Such binaries are sure candidates for the  Laser Interferometer Space Antenna (LISA) mission.
My work was presented at the \href{https://aas.org/meetings/aas237}{\color{C2}237th American Astronomical Society meeting}.


\textbf{Studying primordial gravitational waves from inflation and reheating phase} \hfill Aug 2021 - Present \\
\emph{Advisor: \href{https://physics.iitm.ac.in/~sriram/index.html}{Prof. L. Sriramkumar , IIT Madras}}
\vspace{0.1 cm}\\
Studying the evolution of primordial gravitational waves during the inflationary era data and the reheating phase of the universe. 






\textbf{Signal detection and parameter estimation using LIGO O1 and O2 data}\hfill May 2019 - Jul 2019 \\
\emph{Advisor: \href{https://www.iiserkol.ac.in/~rajesh/}{Prof. Rajesh Nayak , IISER Kolkata}}
\vspace{0.1 cm}\\
The project involved learning the basics of gravitational waves data analysis and parameter estimation using LIGO's publicly available data from O1 and O2 run.




\begin{comment}



%----------------------------------------------------------------------------------------
%----------------------------------------------------------------------------------------
%	RELEVANT COURSE SECTION
%----------------------------------------------------------------------------------------

\section{Projects}
\begin{itemize}[leftmargin=*]
	\item \textbf{Aditya Vijaykumar}, MV Saketh, Sumit Kumar, Parameswaran Ajith, Tirthankar Roy Choudhury.\\
	\textit{Probing the cosmological large-scale structure using gravitational-wave observations} \\
	\textit{(manuscript under LIGO PnP review, to be submitted to arXiv soon)}
	\\
	\item \textbf{Aditya Vijaykumar}, Shasvath Kapadia, Parameswaran Ajith.\\
	\textit{Constraining the time-variation of the Gravitational constant using gravitational-wave observations of binary neutron stars} \\
	\textit{(manuscript under LIGO PnP review, to be submitted to arXiv soon)}
	\\
	\item \textbf{Aditya Vijaykumar}, Nathan Johnson-McDaniel, Rahul Kashyap, Arunava Mukherjee, Parameswaran Ajith.\\	
	\textit{Constraints on Black Hole Mimickers from the Gravitational-wave Transient Catalog (GWTC) -1 }
	\\
	\item Apratim Ganguly, \textbf{Aditya Vijaykumar}, Abhirup Ghosh, Parameswaran Ajith.\\	
	\textit{Probing General Relativity from the consistency of inspiral and merger-ringdown of Binary Black Holes}


\end{itemize}



\end{comment}

% \section{Papers}
% \begin{etaremune}

% 	\item Saleem et al. (including \textbf{Aditya Vijaykumar})\\
% 	\textit{The Science Case for LIGO-India}\\
% 	\href{https://arxiv.org/abs/2105.01716}{arXiv:2105.01716}.

% 	\item Abbott et al. (LIGO Scientific and Virgo Collaborations, including \textbf{Aditya Vijaykumar})\\
% 	\textit{Tests of General Relativity with Binary Black Holes from the second LIGO-Virgo Gravitational-Wave Transient Catalog.}\\
% 	\href{https://arxiv.org/abs/2010.14529}{arXiv:2010.14529}.	
	
% 	\item Abbott et al. (LIGO Scientific and Virgo Collaborations, including \textbf{Aditya Vijaykumar})\\
% 	\textit{	GWTC-2: Compact Binary Coalescences Observed by LIGO and Virgo During the First Half of the Third Observing Run}.\\
% 	\href{https://arxiv.org/abs/2010.14527}{arXiv:2010.14527}.
	
% 	\item 
% 	\textbf{Aditya Vijaykumar}, M.~V.~S.~Saketh, Sumit Kumar, Parameswaran Ajith, Tirthankar Roy Choudhury\\
% 	\textit{Probing the large scale structure using gravitational wave observations of binary black holes},\\
% 	Submitted to \textit{Physical Review Letters}, \href{https://arxiv.org/abs/2005.01111}{arXiv:2005.01111}.\\
% 	\textit{In press}: \href{https://astrobites.org/2020/05/07/binary-black-holes-tangled-up-in-the-cosmic-web/}{Astrobites}.
	
% 	\item 
% 	\textbf{Aditya Vijaykumar}, Shasvath J. Kapadia, Parameswaran Ajith\\
% 	\textit{Constraints on the time variation of the gravitational constant using gravitational wave observations of binary neutron stars},\\
% 	\href{https://journals.aps.org/prl/abstract/10.1103/PhysRevLett.126.141104}{\textit{Phys. Rev. Lett}. 126, 141104 (2021)}, \href{https://arxiv.org/abs/2003.12832}{arXiv:2003.12832}.\\
% 	\textit{In press}: \href{https://phys.org/news/2021-05-constraints-variation-gravitational-constant.html}{phys.org}.
	
% 	\item 
% 	P.~Virtanen {\it et al.} (including \textbf{Aditya Vijaykumar} as \textit{SciPy 1.0 Contributor})\\
% 	\textit{SciPy 1.0--Fundamental Algorithms for Scientific Computing in Python},\\
% 	\href{https://www.nature.com/articles/s41592-019-0686-2}{\textit{Nat Methods} 17, 261–272 (2020)},
% 	\href{https://arxiv.org/abs/1907.10121}{arXiv:1907.10121}.
% \end{etaremune}

% \section{Seminars and Invited Talks}
% \begin{itemize}[leftmargin=*]
% 	 \item \textit{Probing Large Scale Structure using Binary Black Hole Observations} at \textbf{\textit{Instituut-Lorentz} for Theoretical Physics, Leiden University}, Leiden, Netherlands, June 2020 (Online)
% 	 \item \textit{Probing Large Scale Structure using Binary Black Hole Observations} at \textbf{The Inter-University Centre for Astronomy and Astrophysics (IUCAA)}, Pune, India, September 2019
% 	 \item \textit{Probing Large Scale Structure using Binary Black Hole Observations} at \textbf{Max Planck Institute for Gravitational Physics}, Hannover, Germany, June 2019
% \end{itemize}
\section{Publications}
\begin{itemize}[leftmargin=*]
    %\item (In preperation) \textbf{Pranav Satheesh},Chandra Kant Mishra\\
    %\textit{Ready-to-use eccentric frequency domain templates with non quadrapole} modes
    \item (In preperation) Tamal RoyChowdhury, Abhishek Chattaraj, \textbf{Pranav Satheesh}, Chandra Kant Mishra\\
    \textit{Eccentric time domain and frequency domain Inspiral-Merger-Ringdown hybrid waveforms}
\end{itemize}



\section{Presentations and Posters}

% \textbf{Contributed talks and posters}\\
\begin{itemize}[leftmargin=*]
    \item Tamal RoyChowdhury, Abhishek Chattaraj, \textbf{Pranav Satheesh}, Chandra Kant Mishra, \href{https://www.amaldi14.org/}{\textbf{14th Amaldi 2021, 19-23 July (online)}}, \href{https://drive.google.com/file/d/1tCGgniOafmLfrbhD4X7ca3VkhOJtOSc2/view?usp=sharing}{\color{C2} \textit{Elements of modelling binary black holes in eccentric orbits through inspiral, merger and ringdown stages}}\\
    
    \item Tamal RoyChowdhury, Abhishek Chattaraj, \textbf{Pranav Satheesh}, Chandra Kant Mishra, \href{http://kiw8.org/program/}{\textbf{8th KAGRA International Workshop, 2021}}, \href{http://kiw8.org/data/users/348932af5b3e54a8958255b121ccef68/poster/50_pranavsatheesh-poster.pdf}{\color{C2}{\textit{Modelling Frequency Domain Inspiral-merger-ringdown Wave-forms for Eccentric Binary Black Hole Mergers}}}
    
    \item \textbf{Pranav Satheesh}, Prasenjit Saha, Hans Martin Schmid , \href{https://aas.org/meetings/aas237}{\textbf{237th American Astronomical Society meet, 2021}}, \href{aas237-aas.ipostersessions.com/Default.aspx?s=79-64-0C-43-B0-53-8B-48-C7-A1-41-CE-DF-A9-70-2A}{\color{C2}\textit{A spectropolarimetric method for predicting the gravitational wave polarization of LISA verification binaries}}
    
    \item  \textbf{Pranav Satheesh}, \href{https://ras.ac.uk/ras-2020-posters}{\textbf{RAS Career Poster Exhibition, 2020}}, \href{https://ras.ac.uk/poster-contest/pranav-satheesh}{\color{C2} {\textit{Frequency Domain Gravitational Waveform Modelling for Eccentric Black Hole Binaries}}}\\ 
    

\end{itemize}   
% \textbf{Attended meetings}\\
% \begin{itemize}[leftmargin=*]
%     \item BitGrav meeting %add poster link
%     \item CMU meeting
% \end{itemize}

\section{Scholarships and Awards}
\begin{itemize}[leftmargin=*]

 \item Selected among top 8 students in India for \href{https://swissnex.org/india/thinkswiss/}{ThinkSwiss Research Scholarship} \hfill 2020

 \item Recepient of the \href{http://www.inspire-dst.gov.in/scholarship.html}{INSPIRE-DST Scholarship for Higher Education}  \hfill 2017 - \textit{Present}
\end{itemize}


\section{Professional Memberships}
\begin{itemize}[leftmargin=*]
    \item \textit{Member}, \textbf{\color{C3} LIGO Scientific Collaboration} \hfill 2021 -   \textit{Present}
    \item \textit{Undergraduate Member}, \textbf{\color{C3} American Astronomical Society} \hfill 2020-\textit{2021}\\
\end{itemize}

\section{Teaching Experience}
\begin{itemize}[leftmargin=*]
    \item {\color{C2} Teaching Assistant}, \textbf{\color{C3}Complex Networks (ID5080)} \hfill Aug 2021 - \textit{Present} \\
    \emph{Graduate level course at IIT Madras} 
    \item {\color{C2} Teaching Assistant}, \href{https://semaphorep.github.io/codeastro/}{\textbf{Code Astro 2021}}   \hfill June 2021  \\
    \emph{Virtual Software Engineering Workshop for Astronomy supported by \\ the Heising-Simons Foundation.}
\end{itemize}

\section{Schools and Workshops}
\begin{itemize}[leftmargin=*]
    \item \href{https://youtu.be/zXDrQ_-WNUg}{\textbf{LISC Continous Gravitational Wave Workshop}} (Online) \hfill Oct 2021
    \item \href{https://sites.psu.edu/paxvii/}{\textbf{Physics and Astrophysics at the Extreme (PAX-VII) Workshop}} (Online) \hfill Aug 2021
    %\item Participant, \textbf{2021 Sagan Exoplanet Summer Virtual Workshop}, NASA Exoplanet Science Institute, California Institute of Technology, July 2021 
	\item \href{https://www.icts.res.in/program/gws2021}{\textbf{ICTS Summer School on Gravitational Wave Astronomy}} (Online) \hfill Jul 2021 
	\item \href{http://ipta4gw.org/meetings/2021/}{\textbf{IPTA Student Workshop}} (Online) \hfill June 2021 
	\item \href{https://icerm.brown.edu/programs/sp-f20/w2/#workshopparticipants}{\textbf{Mathematical and Computational Approaches for solving \\ source-free Einstein Field Equations}} ICERM, Brown University (online) \hfill Oct 2020
	\item \href{https://www.icts.res.in/program/peu}{\textbf{Physics of the Early Universe}}, ICTS (Online)\hfill  Sep 2020 
	\item \href{https://www.icts.res.in/program/gws2020}{\textbf{ICTS Summer School on Gravitational Wave Astrophysics}} \hfill May 2020 
\end{itemize}

\section{Relevant Coursework}
General Relativity and Cosmology, Advanced General Relativity, Methods of Computational Physics, Numerical Methods and Programming lab, Classical Field Theory, Advanced Particle Physics, High Energy Physics, Statistical Physics, Quantum Mechanics, Classical Mechanics, 
Mathematical Physics, Differential Equations

% \section{Outreach Talks}
% \begin{itemize}[leftmargin=*]
% 	\item  \textit{The Whats, Whys and Hows of Gravitational-wave Astronomy}, \textbf{BMS College of Engineering, Bengaluru}, November 2019
% 	\item \textit{Gravitational Waves - A New Tool for Cosmology!} at \textbf{Vigyan Samagam}, Visvesvaraya Industrial and Technological Museum, Bengaluru, India, August 2019

% \end{itemize}




% %----------------------------------------------------------------------------------------
% %	TECHNICAL SKILLS SECTION
% %----------------------------------------------------------------------------------------





\section{Technical \hspace{2mm} Skills}
\textbf{Programming Languages} - Python, C, C++, Shell script\\
\textbf{Softwares} - Mathematica, SAO DS9 \\
\textbf{Tools/Frameworks} - \LaTeX, Git



\section{Outreach}

\textbf{Service}
\begin{itemize}[leftmargin=*]
    \item Head, \href{https://horizoniitm.github.io/horizon/}{\textbf{Horizon: The Physics and Astronomy Club of IIT Madras}} \hfill 2019-2020\\
     I headed the student run physics and astronomy club at IIT Madras under the Center of Innovation (CFI).
    We engage the student community in the campus though various projects, lectures, workshops and compeetitive events.
\end{itemize}
\textbf{Articles}
\begin{itemize}[leftmargin=*]
    \item Undergraduate Research summary in \href{https://astrobites.org/2021/06/20/ur-a-spectropolarimetric-method-for-predicting-the-gravitational-wave-polarisation-of-lisa-verification-binaries/}{Astrobites} \\ \textit{\color{C2} UR: A spectropolarimetric method for predicting the gravitational wave polarisation of LISA verification }
\end{itemize}

\textbf{Talks}
\begin{itemize}[leftmargin=*]
    \item \textit{\color{C2} Python for Astronomy}, An \href{https://youtu.be/HfYR0uwYAyM}{Youtube lecture series} offered by me as part of Horizon \hfill Jul 2020
    \item \textit{\color{C2} Relativity and Gravitation}, \href{https://github.com/HorizonIITM/summer-school-2021}{Horizon-IITM Summer School} \hfill July 2021
    \item Tutor, \textit{\color{C2} Analysis of Globular Clusters Using
    Colour-Magnitude Diagrams}, Shaastra IITM \hfill Jan 2020
\end{itemize}





%\textbf{Talks}
% \begin{itemize}
%     \item GW
% \end{itemize}

% \section{References}
% \begin{itemize}[leftmargin=*]
%  \item Prof. Parameswaran Ajith, ICTS -- \href{mailto:ajith@icts.res.in}{ajith@icts.res.in}
%  \item Dr. Shasvath Kapadia, ICTS -- \href{mailto:shasvath.kapadia@icts.res.in}{shasvath.kapadia@icts.res.in}
%  \item Dr. Sumit Kumar, AEI Hannover -- \href{mailto:sumit.kumar@aei.mpg.de}{sumit.kumar@aei.mpg.de}
%  \item Prof. Bala Iyer, ICTS -- \href{mailto:bala.iyer@icts.res.in}{bala.iyer@icts.res.in}
% \end{itemize}

\end{resume}
\end{document}
