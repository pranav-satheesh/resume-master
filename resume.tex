%----------------------------------------------------------------------------------------
%	PACKAGES AND OTHER DOCUMENT CONFIGURATIONS
%----------------------------------------------------------------------------------------
\documentclass[margin, centered]{res}
\topmargin=-0.5in
\oddsidemargin -.5in
\evensidemargin -.5in
\textwidth=6.5in
\itemsep=0in
\parsep=0in
\newsectionwidth{1in}
\usepackage[pdftex]{graphicx}
\usepackage{etaremune}
\usepackage{enumitem}
\usepackage{wrapfig}
\usepackage[dvipsnames]{xcolor}
\usepackage{helvet}
\usepackage{comment}
\usepackage{fancyhdr}

\usepackage[colorlinks = true,
            linkcolor = BrickRed,
            urlcolor  = BrickRed,
            citecolor = BrickRed,
            anchorcolor = BrickRed]{hyperref}
\setlength{\textwidth}{6.5in} % Text width of the document
\setlength{\textheight}{720pt}

\begin{document}

%----------------------------------------------------------------------------------------
%	NAME AND ADDRESS SECTION
%----------------------------------------------------------------------------------------\
\begin{center}
    \hspace{-\hoffset}
    \huge {\textcolor{black}{\textbf{Pranav Satheesh}}}
\end{center}

\vspace{-5mm}
\moveleft\hoffset\vbox{\hrule width 19cm height 0.5pt}
\vspace{-8mm}
\begin{center}
    \hspace{-\hoffset}
    \href{mailto:pranavsatheesh17@gmail.com}{pranavsatheesh17@gmail.com} ~\textbullet~ \href{https://pranav-satheesh.github.io/}{pranav-satheesh.github.io} ~\textbullet~ \ Indian Institute of Technology, Madras, India
\end{center}
\vspace{-7mm}
\begin{resume}

%----------------------------------------------------------------------------------------
%	EDUCATION SECTION
%----------------------------------------------------------------------------------------
\section{Reserach Interests}
Gravitational Wave Astronomy and Astrophysics, Post-Newtonian theory and Numerical Relativity, Cosmology using Gravitational waves
\\
\section{Education}


\textbf{\href{https://www.iitm.ac.in/}{Indian Institute of Technology Madras, Chennai, India}} \hfill 2017 - 2022 (expected) \\
\textbf{BS-MS Dual Degree Physics} \\
CGPA: 9.19/10
%----------------------------------------------------------------------------------------
%	EXPERIENCE SECTION
%----------------------------------------------------------------------------------------
\section{Research Experience}


\textbf{Improving eccentric binary Black hole models by including spin effects} \hfill Jul 2021 - \textit{Present}\\
\emph{Mentored by \href{https://physics.iitm.ac.in/ckm}{Dr. Chandra Kant Mishra, IIT Madras} and \href{https://www.icts.res.in/people/prayush-kumar}{Dr. Prayush Kumar, ICTS}}

\begin{itemize}
    \item This work is part of my final year thesis. I'm working on improving a non-spinning eccentric black hole waveform model (ENIGMA) by including effects of Black Hole spins in the waveform.
\end{itemize}

\textbf{Constructing ready-to-use frequency domain waveform model for eccentric binary black holes including non-quadrupole modes} \\
\emph{Mentored by \href{https://physics.iitm.ac.in/ckm}{Dr. Chandra Kant Mishra, IIT Madras}}\hfill Aug 2019 - August 2021

\begin{itemize}
    \item I worked on producing an efficient, ready-to-use frequency domain waveform model for eccentric binary black holes. The waveform also accounts for periastron effects 
    \item The waveform is produced by applying \emph{Stationary Phase approximation} on time domain waveforms that includes non-quadrupole modes.
    \item My work was presented at \href{http://kiw8.org/program/}{8th KAGRA International Workshop}  and \href{https://www.amaldi14.org/}{14th Edoardo Amaldi Conference}.
\end{itemize}

\textbf{Studying primordial gravitational waves from inflation and reheating phase} \\
\emph{Mentored by \href{https://physics.iitm.ac.in/~sriram/index.html}{Prof. L. Sriramkumar , IIT Madras}}\hfill Aug 2021 - Present
\begin{itemize}
    \item I'm studying the evolution of primordial gravitational waves during the inflationary era data and the reheating phase of the universe. 
\end{itemize}


\textbf{Polarimetric method for predicting gravitational wave polarization of LISA verification binaries} \\
\emph{Mentored by \href{https://www.ctac.uzh.ch/en/Research/research-groups/Prasenjit-Saha.html}{Prof. Prasenjit Saha, University of Zurich} }\hfill Summer 2020 

\begin{itemize}
    \item I worked on developing a method utilizing Polarimetry to measure the orientation and inclination of the binary system (HP Lib). Such binaries are sure candidates for the  Laser Interferometer Space Antenna (LISA) mission.
    \item My work was presented at the \href{https://aas.org/meetings/aas237}{237th American Astronomical Society meeting}.
\end{itemize}


\textbf{Signal detection and parameter estimation using LIGO O1 and O2 data} \\
\emph{Mentored by \href{https://www.iiserkol.ac.in/~rajesh/}{Prof. Rajesh Nayak , IISER Kolkata}}\hfill Summer 2019
\begin{itemize}
    \item The project involved learning the basics of gravitational waves data analysis and parameter estimation using LIGO's publicly available data from O1 and O2 run.
\end{itemize}



\begin{comment}



%----------------------------------------------------------------------------------------
%----------------------------------------------------------------------------------------
%	RELEVANT COURSE SECTION
%----------------------------------------------------------------------------------------

\section{Projects}
\begin{itemize}[leftmargin=*]
	\item \textbf{Aditya Vijaykumar}, MV Saketh, Sumit Kumar, Parameswaran Ajith, Tirthankar Roy Choudhury.\\
	\textit{Probing the cosmological large-scale structure using gravitational-wave observations} \\
	\textit{(manuscript under LIGO PnP review, to be submitted to arXiv soon)}
	\\
	\item \textbf{Aditya Vijaykumar}, Shasvath Kapadia, Parameswaran Ajith.\\
	\textit{Constraining the time-variation of the Gravitational constant using gravitational-wave observations of binary neutron stars} \\
	\textit{(manuscript under LIGO PnP review, to be submitted to arXiv soon)}
	\\
	\item \textbf{Aditya Vijaykumar}, Nathan Johnson-McDaniel, Rahul Kashyap, Arunava Mukherjee, Parameswaran Ajith.\\	
	\textit{Constraints on Black Hole Mimickers from the Gravitational-wave Transient Catalog (GWTC) -1 }
	\\
	\item Apratim Ganguly, \textbf{Aditya Vijaykumar}, Abhirup Ghosh, Parameswaran Ajith.\\	
	\textit{Probing General Relativity from the consistency of inspiral and merger-ringdown of Binary Black Holes}


\end{itemize}



\end{comment}

% \section{Papers}
% \begin{etaremune}

% 	\item Saleem et al. (including \textbf{Aditya Vijaykumar})\\
% 	\textit{The Science Case for LIGO-India}\\
% 	\href{https://arxiv.org/abs/2105.01716}{arXiv:2105.01716}.

% 	\item Abbott et al. (LIGO Scientific and Virgo Collaborations, including \textbf{Aditya Vijaykumar})\\
% 	\textit{Tests of General Relativity with Binary Black Holes from the second LIGO-Virgo Gravitational-Wave Transient Catalog.}\\
% 	\href{https://arxiv.org/abs/2010.14529}{arXiv:2010.14529}.	
	
% 	\item Abbott et al. (LIGO Scientific and Virgo Collaborations, including \textbf{Aditya Vijaykumar})\\
% 	\textit{	GWTC-2: Compact Binary Coalescences Observed by LIGO and Virgo During the First Half of the Third Observing Run}.\\
% 	\href{https://arxiv.org/abs/2010.14527}{arXiv:2010.14527}.
	
% 	\item 
% 	\textbf{Aditya Vijaykumar}, M.~V.~S.~Saketh, Sumit Kumar, Parameswaran Ajith, Tirthankar Roy Choudhury\\
% 	\textit{Probing the large scale structure using gravitational wave observations of binary black holes},\\
% 	Submitted to \textit{Physical Review Letters}, \href{https://arxiv.org/abs/2005.01111}{arXiv:2005.01111}.\\
% 	\textit{In press}: \href{https://astrobites.org/2020/05/07/binary-black-holes-tangled-up-in-the-cosmic-web/}{Astrobites}.
	
% 	\item 
% 	\textbf{Aditya Vijaykumar}, Shasvath J. Kapadia, Parameswaran Ajith\\
% 	\textit{Constraints on the time variation of the gravitational constant using gravitational wave observations of binary neutron stars},\\
% 	\href{https://journals.aps.org/prl/abstract/10.1103/PhysRevLett.126.141104}{\textit{Phys. Rev. Lett}. 126, 141104 (2021)}, \href{https://arxiv.org/abs/2003.12832}{arXiv:2003.12832}.\\
% 	\textit{In press}: \href{https://phys.org/news/2021-05-constraints-variation-gravitational-constant.html}{phys.org}.
	
% 	\item 
% 	P.~Virtanen {\it et al.} (including \textbf{Aditya Vijaykumar} as \textit{SciPy 1.0 Contributor})\\
% 	\textit{SciPy 1.0--Fundamental Algorithms for Scientific Computing in Python},\\
% 	\href{https://www.nature.com/articles/s41592-019-0686-2}{\textit{Nat Methods} 17, 261–272 (2020)},
% 	\href{https://arxiv.org/abs/1907.10121}{arXiv:1907.10121}.
% \end{etaremune}

% \section{Seminars and Invited Talks}
% \begin{itemize}[leftmargin=*]
% 	 \item \textit{Probing Large Scale Structure using Binary Black Hole Observations} at \textbf{\textit{Instituut-Lorentz} for Theoretical Physics, Leiden University}, Leiden, Netherlands, June 2020 (Online)
% 	 \item \textit{Probing Large Scale Structure using Binary Black Hole Observations} at \textbf{The Inter-University Centre for Astronomy and Astrophysics (IUCAA)}, Pune, India, September 2019
% 	 \item \textit{Probing Large Scale Structure using Binary Black Hole Observations} at \textbf{Max Planck Institute for Gravitational Physics}, Hannover, Germany, June 2019
% \end{itemize}
\section{Publications}
\begin{itemize}[leftmargin=*]
    \item (In preperation) \textbf{Pranav Satheesh},Chandra Kant Mishra\\
    \textit{Ready-to-use eccentric frequency domain templates with non quadrapole} modes
    \item (In preperation) Tamal RoyChowdhury, Abhishek Chattaraj, \textbf{Pranav Satheesh}, Chandra Kant Mishra\\
    \textit{Eccentric time domain and frequency domain Inspiral-Merger-Ringdown hybrid waveforms}
\end{itemize}



\section{Conferences}

% \textbf{Contributed talks and posters}\\
\begin{itemize}[leftmargin=*]
    \item \href{https://www.amaldi14.org/}{\textbf{14th Edoardo Amaldi Conference on Gravitational Waves, 2021}}\\
    Tamal RoyChowdhury, Abhishek Chattaraj, \textbf{Pranav Satheesh}, Chandra Kant Mishra\\
    \textit{Elements of modelling binary black holes in eccentric orbits through inspiral, merger and ringdown stages}
    
    \item \href{http://kiw8.org/program/}{\textbf{8th KAGRA International Workshop, 2021}}\\
    Tamal RoyChowdhury, Abhishek Chattaraj, \textbf{Pranav Satheesh}, Chandra Kant Mishra\\
    \href{http://kiw8.org/data/users/348932af5b3e54a8958255b121ccef68/poster/50_pranavsatheesh-poster.pdf}{\textit{Modelling Frequency Domain Inspiral-merger-ringdown Wave-forms for Eccentric Binary Black Hole Mergers}}
    
    \item \href{https://aas.org/meetings/aas237}{\textbf{237th American Astronomical Society meet, 2021}}\\
    \textbf{Pranav Satheesh}, Prasenjit Saha, Hans Martin Schmid \\
    \href{aas237-aas.ipostersessions.com/Default.aspx?s=79-64-0C-43-B0-53-8B-48-C7-A1-41-CE-DF-A9-70-2A}{\textit{A spectropolarimetric method for predicting the gravitational wave polarization of LISA verification binaries}}
    
     \item \href{https://ras.ac.uk/ras-2020-posters}{\textbf{RAS Career Poster Exhibition, 2020}}\\
    \textbf{Pranav Satheesh} \\
    \href{https://ras.ac.uk/poster-contest/pranav-satheesh}{\textit{Frequency Domain Gravitational Waveform Modelling for Eccentric Black Hole Binaries}}\\ 
    

\end{itemize}   
% \textbf{Attended meetings}\\
% \begin{itemize}[leftmargin=*]
%     \item BitGrav meeting %add poster link
%     \item CMU meeting
% \end{itemize}

\section{Scholarships and Awards}
\begin{itemize}[leftmargin=*]

 \item Selected for \textbf{ThinkSwiss Research Scholarship} by Swissnex, India

 \item Recepient of the \href{http://www.inspire-dst.gov.in/scholarship.html}{INSPIRE-DST Scholarship for Higher Education} for the period 2017 to 2021
\end{itemize}


\section{Schools and Workshops}
\begin{itemize}[leftmargin=*]

    \item Participant, \textbf{2021 Sagan Exoplanet Summer Virtual Workshop}, NASA Exoplanet Science Institute, California Institute of Technology, July 2021 
	\item Participant, \textbf{ICTS Summer School on Gravitational Wave Astronomy}, ICTS, Bengaluru, India, July 2021 (Online)
	\item Tutor, \textbf{Code Astro 2021}, June 2021 (Online)
	\item Participant, ICERM, Brown University (online)
	\item Participant, Bilby workshop, ICTS 
	\item Participant, \textbf{Physics of the Early Universe}, ICTS, Bengaluru, India, September 2020 (Online)
	\item Participant, \textbf{ICTS Summer School on Gravitational Wave Astronomy}, ICTS, Bengaluru, India, May-June 2020 (Online)
	\item Participant, \textbf{Code Astro 2020}, June 2020 (Online)
\end{itemize}
% \section{Outreach Talks}
% \begin{itemize}[leftmargin=*]
% 	\item  \textit{The Whats, Whys and Hows of Gravitational-wave Astronomy}, \textbf{BMS College of Engineering, Bengaluru}, November 2019
% 	\item \textit{Gravitational Waves - A New Tool for Cosmology!} at \textbf{Vigyan Samagam}, Visvesvaraya Industrial and Technological Museum, Bengaluru, India, August 2019

% \end{itemize}




% %----------------------------------------------------------------------------------------
% %	TECHNICAL SKILLS SECTION
% %----------------------------------------------------------------------------------------

\section{Relevant Coursework}
\begin{itemize}[leftmargin=*]
    \item General Relativity and Cosmology, Advanced General Relativity, Classical Field Theory, Advanced Particle Physics, High Energy Physics, Computational Physics, Advanced Statistical Physics, Quantum Mechanics, Classical Mechanics
\end{itemize}



\section{Technical \hspace{2mm} Skills}
\textbf{Programming Languages} - Python, C, C++\\
\textbf{Softwares} - Mathematica, SAO DS9 \\
\textbf{Tools/Frameworks} - \LaTeX, Git

\section{Professional Memberships}
\textit{Undergraduate Member}, \textbf{American Astronomical Society}\\

\section{Outreach Talks}

% \textbf{Talks}
% \begin{itemize}
%     \item GW
% \end{itemize}

% \section{References}
% \begin{itemize}[leftmargin=*]
%  \item Prof. Parameswaran Ajith, ICTS -- \href{mailto:ajith@icts.res.in}{ajith@icts.res.in}
%  \item Dr. Shasvath Kapadia, ICTS -- \href{mailto:shasvath.kapadia@icts.res.in}{shasvath.kapadia@icts.res.in}
%  \item Dr. Sumit Kumar, AEI Hannover -- \href{mailto:sumit.kumar@aei.mpg.de}{sumit.kumar@aei.mpg.de}
%  \item Prof. Bala Iyer, ICTS -- \href{mailto:bala.iyer@icts.res.in}{bala.iyer@icts.res.in}
% \end{itemize}

\end{resume}
\end{document}
